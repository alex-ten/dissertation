% I have one comment about the part where you speak about different kinds of uncertainty. An important distinction/concept in the litterature is the difference
% between epistemic and aleatoric uncertainty (e.g. https://proceedings.neurips.cc/paper/2017/file/2650d6089a6d640c5e85b2b88265dc2b-Paper.pdf or https://d-nb.info/1233120980/34)

% One place where you could talk about this distinction is when you discuss Friston'free energy principle, which enable to focus on epistemic uncertainty

% Another comment: you speak of "forward" and "backward" curiosity, but do not define it. It would be useful to do so.

% Another comment: in the curiosity and interest section, you do not discuss so much the issue of time scale. Most works you describe in neuroscience are about short-time scale information seeking. But the longer time scale, like what we studied in the NatCom paper, could also be discussed.


%************************************************
\chapter{The psychology of information-seeking}\label{ch:psychology}
%************************************************

\begin{flushright}{\slshape
    Where is the wisdom we have lost in knowledge? \\
    Where is the knowledge we have lost in information? \\
    --- T. S. Eliot \cite{eliot_rock_2014}}
\end{flushright}
The above quote from T.S. Eliot is one of the earliest sources that inspire the famous Data$\neq$Information$\neq$Knowledge$\neq$Wisdom model (DIKW) -- a framework that draws distinctions and defines relationships between the concepts of data, information, knowledge, and wisdom \cite{sharma_5_2008}. While the DIKW model is tossed around in the context of business administration, distinguishing between information and knowledge is also important for the scientific study of information-seeking. This chapter reviews the current literature (mostly) about non-instrumental information-seeking and presents a theoretical perspective arguing that what makes information inherently valuable for curious beings like humans is its implications for knowledge (and perhaps, wisdom).

\section{Information and information-seeking}

% Senses gather information
Sense organs are ubiquitous in the living world. Indeed, they are so useful that not only animals but plants, fungi, and many other organisms have evolved them \cite{trewavas_plant_2005,braunsdorf_fungal_2016,bourret_census_2006,schwab_evolution_2018}. Moreover, humans endow their tools with artificial sensors to make them more responsive and thus "smarter". %[a picture with sensors would be nice]. 
It is no mystery why sensors are so powerful: they \emph{inform} their owners about what is "out there" so that they can adapt their behavior accordingly. Information, then, is the basis of intelligent behavior. It is also a central concept in this thesis, so we need to establish what it means more precisely. 

% Three stances on information
Information is an important concept across many disciplines. Even though precise definitions differ, it is possible to identify three major conceptual stances on what information means \cite{adriaans_introduction_2008}. According to one perspective, information refers to declarative descriptions of the mentally represented world, which can be obtained, for example, through empirical observation, linguistic communication, or "armchair" deduction. This is the sense in which the word 'information' tends to be used in lay conversation. Another view characterizes information in terms of uncertainty. Here, information is viewed as an abstract communication process by which uncertainty about some random event can be reduced. This formulation is commonly adopted in Information Theory \cite{shannon_mathematical_1948}. Finally, there is an approach that treats information as the complexity of the simplest possible representation of an object in a given (i.e., fixed) system. This is a "Kolmogorov-complexity" stance on information \cite{kolmogorov1965three}. It underlies the intuition that simple objects require less information to be described in, say, natural language or neural code, compared to complex objects.

% All three stances are present in sensing
While these three perspectives may seem quite far apart, they are demonstrably and rather intricately connected \cite{adriaans_introduction_2008}. This is particularly clear when we consider how these stances converge within a single information-processing sequence of events, such as sensing. Through senses and neural substrates, organisms can systematically represent entities in the environment (stance 3) thereby reducing uncertainty about presence or absence of particular stimuli (stance 2) and \emph{forming} an \emph{in}ternal description of the surrounding world (stance 1). In other words, sensing is a process of representing, communicating, and interpreting information\footnote{Mnemonic retrieval of information can be viewed as a kind of "internal sensing" by which one system "observes" and/or encodes information from another, all within the same brain.}

% Active sensing = information-seeking as a solution for too much information
Since the world is so incredibly complex, there is probably more potentially observable and thinkable information in it then organisms can possibly represent and process \cite{kolmogorov1965three}. Given the limited computational resources, the overabundance of potential information in the world means that it needs to be somehow funneled for senses to be useful \cite{gottlieb_towards_2018}. Physical features of sense organs responsible for domain specialization (e.g., light, sound, pressure) and sensitivity to specific intensity ranges within domains \cite[e.g.,][]{schwab_evolution_2018} can be viewed as passive, structural funnels of information. Another, active kind of funneling is enabled by the ability to selectively expose one's external and internal sensors to specific stimulation via actions such as movement \cite[e.g. ][]{gottlieb_information-seeking_2013} and neuromodulation \cite[e.g.,][]{yu_uncertainty_2005}. This \emph{active} sensing behavior can be characterized as \emph{information-seeking}. How do organisms control their behavior to expose themselves to "good" information?

\section{Value of information}

% Information-seeking must be deployed strategically to increase biological fitness
Information is beneficial only insofar as it communicates what the organisms should care about. Thus, information-seeking must be deployed strategically and ultimately optimize the biological fitness of a species. Individual organisms, of course, do not know how to optimize this global fitness function directly \cite{ten_berge_procedural_1999,singh_intrinsically_2010,gottlieb_information-seeking_2013}. Instead, phenotypes that come to implement biologically advantageous behavioral tendencies emerge through evolution and ontogenetic development. Information-seeking is likely to be one such tendency, but it is not always obvious how information that organisms seek contributes to the biological fitness of their species.

% % Instrumental/non-instrumental distinction (new)
It is easy to see why information-seeking is biologically advantageous in \emph{instrumental} contexts. Instrumental information-seeking includes behaviors driven by \emph{extrinsically valuable} states that cannot be achieved by sensory stimulation alone. For example, although foraging for food (or tracking a predator) requires seeking information, the ulterior biologically rewarding end is the consumption of nutrients (or avoidance of predation) -- something that cannot be achieved solely by observing or thinking about food (or predators). However, sometimes, information seems to be sought in the apparent absence of an extrinsically valuable state. Valuable states that lack extrinsic value are \emph{intrinsically} valuable by definition. Information-seeking that is driven by such state is called \emph{non-instrumental}.

% % Instrumental/non-instrumental distinction (old)
% A common approach to organizing information-seeking behaviors is to classify them into two broad categories  \cite{gottlieb_information-seeking_2013,gottlieb_towards_2018}. \emph{Instrumental} information-seeking includes behaviors driven by \emph{extrinsically valued} states that cannot be achieved by sensory stimulation alone. For example, although foraging for food (or tracking a predator) requires seeking information, the ulterior biologically rewarding end is the consumption of nutrients (or avoidance of predation) -- something that cannot be achieved solely by observing or thinking about food (or predators). \graffito{\emph{Rewards} are defined by their function of propagating units of selection (e.g., behaviors or genes). In behavior, the term extends both to certain internally represented information about stimuli and to the actual stimulating objects.} \emph{Non-instrumental} information-seeking includes everything else. While it is possible to identify goals separable from the consumption of information \emph{per se} (e.g., gains in knowledge, feelings of competence, etc.), information is sufficient to achieve such goals. Valuable states that lack extrinsic value are \emph{intrinsically} valuable by definition.

% % Non-instrumental and instrumental information-seeking might share control mechanisms
% Instrumental and non-instrumental information-seeking have some things in common. For example, recent work on the neural mechanisms of attentional priority suggests that circuits commonly implicated in non-instrumental information-sampling underlie attentional control in instrumental contexts (see \cite{gottlieb_curiosity_2020} for a review). This work shows that lateral intraparietal (LIP)\footnote{The homologue of LIP in humans is called "intraparietal lobule" or IPL} neurons in macaque monkeys not only respond to the global uncertainty about potential extrinsic rewards \cite{horan2019parietal} but also spatially encodes an attentional priority map to direct information-sampling saccades towards uncertainty-reducing stimulation \cite{foley_parietal_2017}. Neurons in this map have receptive fields sensitive to expected information-gains of the competing stimuli. Importantly, activity in these intraparietal neurons is unrelated to the expected value of the reward itself or the expected (extrinsic) value of information \cite{horan2019parietal}. This work supports a conclusion that attentional prioritization of competing sources of information relies on the same mechanism in instrumental and non-instrumental contexts.

% Non-instrumental and instrumental information-seeking share reward-processing channels
The most compelling evidence for the intrinsic value of information comes from neuroscience. It has revealed common substrates for processing intrinsically and extrinsically rewarding stimuli. This work has relied on different variants of the "observing task", where a subject decision-maker can choose to observe or forego information about the outcome of a maximally uncertain gamble \cite[reviewed in][]{kidd_psychology_2015,cervera_systems_2020}. One early study showed that consuming a water-reward and observing information about the upcoming water-reward is processed by the same structure in the macaque monkey brain \cite{bromberg-martin_midbrain_2009}. Specifically, the well-studied "reward-signaling" \acf{DA} neurons of the midbrain \cite{schultz_neural_1997} responded to more-than- and less-than-expected amounts of water in the same way they responded to information about the corresponding amounts of water \cite{bromberg-martin_midbrain_2009}. Moreover, some neurons in the \acf{LHb} region encoded \acp{IPE} similarly to how these neurons encoded prediction errors about extrinsic rewards: their activity was increased when less-than-expected information was promised to the subjects and it decreased they were promised more information than they otherwise expected \cite{bromberg-martin_lateral_2011}. In another study, \citeauthor{blanchard_orbitofrontal_2015} et al. found that distinct subpopulations of neurons in the \ac{OFC} orthogonally encoded the potential amount of water-reward and the potential validity of an offered cue, i.e., its "informativeness" \cite{blanchard_orbitofrontal_2015}.

% The brain reinforces non-instrumental information-seeking and models information sources
One of the observing task's many lessons is that the brain processes non-instrumental information similarly to how it deals with information about extrinsically valuable stimuli (e.g., as if it was experiencing water consumption). This suggests that information \emph{about} an upcoming reward and information \emph{from} the reward itself have similar roles. The most salient role of information \emph{from} rewards is to drive learning \cite{schultz_dopamine_2016}. The consumption of extrinsically rewarding stimuli gets registered by the brain which uses the corresponding information to adjust its context-dependent predictions and reinforce or suppress preceding appetitive behavior. The study by Bromberg-Martin and Hikosaka \cite{bromberg-martin_midbrain_2009} shows that the brain also reinforces non-instrumental information-sampling behaviors and maintains a model for predicting when and how much of such information can be expected.

% The brain represents value of non-instrumental information separately and likely tracks it in the global and multidimensional internal-state map in the OFC.
The observation that a subpopulation of neurons in the \ac{OFC} are sensitive specifically to non-instrumental information \cite{blanchard_orbitofrontal_2015} has another important implication. It demonstrates that certain states (or stimuli) can be valuable without conferring any extrinsic benefits. One recent comprehensive account the \ac{OFC}'s function suggests that this brain structure works like a "map" (or, perhaps, more like a GPS tracker) that tells an organism where it is in the so-called \emph{task space} \cite{wilson_orbitofrontal_2014}. Specifically, the \ac{OFC} is proposed to combine multimodal sensory information to represent the organism's current state in relation to the goal(s) of the task at hand. The same sensory context can be used to generate model-based predictions of future states for which the \ac{OFC} can represent their inferred value \cite[see][for details]{stalnaker_what_2015}. Outside the narrow settings of specific laboratory tasks, the OFC might have an even more general function. It might provide a substrate for incorporating incoming sensory information into the process of optimizing policies for the selection of abstract goals \cite{fine2021prefrontal} -- goals that serve transient urges such as biological needs and psychological desires \cite{juechems_where_2019}. These theoretical considerations imply that empirical work by Blanchard et al. \cite{blanchard_orbitofrontal_2015} demonstrates that non-instrumental observations can not only be intrinsically valuable but also be pursued as stand-alone goals.

% The benefits of information -- better planning (part 1)
But how does the intrinsic value of information elevate biological fitness? In the context of the "observing task", advance information about a reward might be useful for arbitrating between alternative goals that the animal might be pursuing, which is important for effective planning in allostatic regulation \cite{fine2021prefrontal,sterling_allostasis_2012}. Note that here, \emph{planning} refers to optimization of action policies (including goal-selection policies; see \cite{fine2021prefrontal}) through inference. For example, knowing that the food is certainly coming enables individuals to focus on tasks other than foraging. This observation hints at a potential explanation for why non-instrumental information might be useful biologically.

% The benefits of information -- better planning (part 2)
Information might contribute to one's knowledge about oneself and the environment. Such knowledge is called \emph{declarative} knowledge, which can be subdivided into \emph{episodic} and \emph{semantic} knowledge categories \cite{tulving_episodic_1972}. One salient function of declarative knowledge is enabling effective planning. The optimization process of planning involves "breaking down" a complex task (or, equivalently, an abstract goal) into simpler (less abstract) ones -- a central principle in hierarchical reinforcement learning \cite{pateria_hierarchical_2021} and sensorimotor control \cite{todorov_optimal_2002}. This suggests that declarative-knowledge acquisition through non-instrumental information-seeking is conducive to robust goal achievement. In fact, \emph{procedural} knowledge -- knowledge of \emph{how} to achieve specific ends -- is proposed to be primary both phylogenetically and ontogenetically \cite{ten_berge_procedural_1999}. Thus, information-seeking (for knowledge-acquisition) might be biologically advantageous because it aids skill acquisition \cite{gottlieb_information-seeking_2013}. When combined with a tendency to pursue arbitrary goals \cite{chu_play_2020}, motivation for competence \cite{deci_concept_2005}, and motivation for communication \cite{boyd_evolution_2018}, valuing non-instrumental information can get especially powerful for the acquisition of large and diverse sets of skills. In fact, competence motivation can put essential constraints on exploratory behaviors since unconstrained exploration runs the risk of spending resources on acquiring information that is practically redundant \cite{baldassarre_intrinsically_2013}. The importance of competence motivation and the resulting goal-exploration is discussed in more detail in \hyperref[ch:ai]{Chapter 3}.

% Transition
The idea that accumulation of useful knowledge increases biological fitness through  non-instrumental information-seeking is intriguing because it provides a biological explanation for phenomena like curiosity, interest, and exploratory play \cite{gottlieb_towards_2018,murayama_process_2019,chu_exploratory_2020}. Artificial intelligence studies that simulate agents with different cognitive features lend some support for this idea (reviewed in \autoref{ch:ai}). However, this idea only describes the phylogenetic mechanism by which organisms can evolve non-instrumental information-seeking behavior. It does not describe the actual features and computational principles enabling knowledge acquisition for an individual. How is non-instrumental information-seeking initiated, sustained, and terminated? The rest of this thesis will revolve around the work -- including my own contributions in \autoref{ch:learning-progress, ch:metacognition} -- aiming to elucidate \emph{how} knowledge acquisition is enabled by intrinsically motivated information-seeking.

\section{Curiosity and interest}

% Introduce 'curiosity' and 'interest'
The terms \emph{curiosity} and \emph{interest} lack universally agreed upon technical definitions for understandable reasons \cite{kidd_psychology_2015,murayama_process_2019,dubey_reconciling_2020}. It is still possible to delineate features of two distinct motivational states that the two labels map onto. For example, an aversive state experienced as deprivation and wanting to resolve one's salient awareness of ignorance can be contrasted with an appetitive state experienced as positive anticipation or the actual enjoyment of learning something new. It is tempting to call the former "curiosity" and the latter "interest", but I will refrain from committing to specific definitions and use these terms rather loosely. Both curiosity and interest play important roles in the continual process by which humans acquire knowledge \cite{murayama_process_2019}. Although they coincide more often than not, motivational states differ in their what triggers them, what neural and behavioral responses ensue, and how they are experiences affectively \cite{day_curiosity_1982,grossnickle_disentangling_2016,hidi_interest_2019,shin_homo_2019,litman_curiosity_2019}.

\subsection{Situational determinants} 

% Different causes of curiosity stem from uncertainty
Throughout history, researchers have proposed seemingly different explanations for what triggers the motivation to seek information in the absence of extrinsic value. These include conflict \cite{berlyne_theory_1954}, ambiguity \cite{ellsberg_risk_1961}, incongruity/dissonance \cite{hunt_experience_1960,festinger_theory_1962}, knowledge-gap \cite{loewenstein_psychology_1994}, unpredictability \cite{shin_homo_2019}, and more \cite[see ][for a review]{loewenstein_psychology_1994,oudeyer_what_2007,bazhydai_curiosity_2020}. A common denominator for all of these proposals seems to be uncertainty. Indeed, uncertainty appears to be a necessary ingredient for sparking curiosity\footnote{I hope you can agree that the idea of a curious omniscient being is oxymoronic.}. One reason for the diversity of propositions is that uncertainty comes in many "shapes" and "sizes".

% Expected and unexpected uncertainty
For instance, Yu and Dayan proposed a qualitative distinction between \emph{expected} and \emph{unexpected} uncertainty \cite{yu_expected_2003,yu_uncertainty_2005}. Expected uncertainty arises when an agent does not hold strong expectations about any particular future outcome or any particular cause of an observed outcome. Unexpected uncertainty arises when an agent observes an outcome or explanation that violates its learned expectations or beliefs. Expected and unexpected uncertainty are processed differently by the brain. Expected uncertainty is associated with the arousal of cholinergic activity resulting in elevated levels of \ac{ACh}, while unexpected uncertainty corresponds to the arousal of the noradrenergic system resulting in higher levels of \ac{NE}.

% Expected and unexpected uncertainty -- do they trigger curiosity?
The status of these two types of uncertainty as triggers of curiosity is yet to be investigated empirically, but if the expected/unexpected taxonomy indeed captures all instances of uncertainty, it stands to reason that either one or both might spark the desire for resolution. There are clues to suggest that both expected and unexpected uncertainty might trigger curiosity. Diminished curiosity was observed in patients with a probable early Alzheimer's disease \cite{daffner_diminished_1992}. These patients spent less time looking at curiosity-inducing stimuli compared to healthy controls. Alzheimer's disease is associated with severe damage to the cholinergic system \cite{ferreira-vieira_alzheimers_2016} involved in the processing of expected uncertainty \cite{yu_uncertainty_2005}. On the other hand, activity in one of the main noradrenergic structures, the \ac{LC}, correlates with heightened arousal, pupil dilation, and more efficient learning \cite{breton-provencher_locus_2021} -- the usual companions of curiosity. Moreover, \ac{LC} interacts with the major DA structure in the \ac{VTA} and can even disseminate dopamine in addition to \ac{NE} across the brain \cite{ranjbar-slamloo_dopamine_2020}. As discussed above, dopamine is involved in reward anticipation and reward processing, and is a major component of the 'wanting' system in the brain \cite{berridge_debate_2007}.

% Unpredictability and incongruity correspond to expected and unexpected uncertainty
Also, pertinent to the current discussion is Shin and Kim's \cite{shin_homo_2019} distinction between two kinds of uncertainty: unpredictability and incongruity. Unpredictability corresponds to states of not knowing "when, where, or how an event has occurred or will occur" \cite[][p. 13]{shin_homo_2019}. On the other hand, incongruity causes backward curiosity, and it arises in situations that violate one's expectations. It should be clear from this characterization that unpredictability and incongruity are similar (if not equivalent) to expected and unexpected uncertainty, respectively.

% Let's clarify some terminology
Whether a person experiences a state of forward or backward curiosity depends on whether information is anticipated "forwardly" or contemplated "backwardly". This implies that while unpredictability and incongruity arise in two different situations, they can fluidly morph into one another. Suppose I am scammed into buying a "loaded" coin that is falsely advertised to turn up heads 99.9\% of the time. If I am genuinely fooled, my expected uncertainty about any future toss of that coin is low. If I then observe 8 out of 10 unexpected tails, not only will I be surprised, but I will also update my belief about the coin so that my expected uncertainty about future tosses will increase. This example illustrates that the term "unexpected uncertainty" conflates surprise \cite{barto_novelty_2013} with the expected uncertainty that ensues. Hence, to avoid confusion, I will abandon the expected/unexpected terminology and refer to unpredictability as uncertainty about an anticipated outcome and incongruity as surprise about an outcome observed in the past.

% Different predictions for uncertainty and surprise
As Shin and Kim \cite{shin_homo_2019} note, unpredictability and incongruity have different relationships with curiosity. Unpredictability has an inverted U-shape relationship with (forward) curiosity \cite{berlyne_theory_1954,day_curiosity_1982,loewenstein_psychology_1994}, while incongruity has a positive monotonic relationship with (backward) curiosity \cite{horstmann_surprise-attention_2015}. The monotonic relationship between incongruity/surprise and curiosity is easier to understand. Surprise signals the inadequacy of one's beliefs, so it makes sense for organisms to minimize it, for example, by being curious and seeking information \cite{schwartenbeck_computational_2019}. This is in line with Friston's \cite{friston_free-energy_2009} free-energy principle -- a mathematical theory of the sustainability of life itself. Surprise minimization is a biological implementation of this principle. Several empirical studies have demonstrated the positive relationship between surprise and curiosity using different behavioral paradigms \cite{berlyne_experimental_1954,itti_bayesian_2009,poli_infants_2020}.

% Curiosity and uncertainty - the inverted U relationship
The prediction of an inverted U-shape relationship between unpredictability/uncertainty and forward curiosity is less intuitively straightforward. In a nutshell, it maintains that prior knowledge determines the subjective intensity of curious states in a nonlinear way: one is predicted to be most motivated to obtain information if one has some incomplete idea as to what this information might be (intermediate uncertainty); conversely, curiosity is predicted to be low when one already has all the relevant knowledge (low uncertainty) or when one has too little knowledge (high uncertainty). The prediction that perfect knowledge should spark no curiosity is trivial, but it is not obvious why very poor knowledge is unlikely to result in curiosity. However, there is considerable evidence supporting this prediction \cite{berlyne_experimental_1954,day_prior_1972,loewenstein_psychology_1994,kang_wick_2009,baranes_eye_2015}.

% Mechanisms of the inverted-U effect
While researchers seem to agree on the general prediction, the proposed mechanisms may vary considerably. For example, Loewenstein's \cite{loewenstein_psychology_1994} proposed mechanism is based on awareness. He argues that poor knowledge prevents an individual from attending to the knowledge gap because information that \emph{is} known is relatively more salient than information that \emph{could} be known. This idea resonates with the Dunning-Kruger effect that describes a tendency of ignorant people to be unaware of their deficiencies in knowledge \cite{dunning_chapter_2011}. Note, however, that this mechanism does not describe an direct link between curiosity and uncertainty. Rather, it shows how the latter relates to the former via the mediating effect of awareness. On the other hand, Berlyne's \cite{berlyne_theory_1954} conflict-based mechanism proposes a direct interaction between uncertainty and curiosity. Conflict is the "disagreement" between competing behavioral/internal responses to a stimulus \cite{berlyne_theory_1954,berlyne_uncertainty_1957}. Highly unfamiliar stimuli fail to arouse conflict because there are no sufficiently activated responses to clash with each other. Curiosity is maximal when there are several competing responses. Berlyne's proposal sits well with Hebb's physiological theory of arousal \cite{hebb_drives_1955}. Specifically, Hebb's notion of \emph{disturbances} in activation patterns of acquired cortical representations -- what he calls phase sequences and cell assemblies, respectively \cite{hebb_organization_2002} -- seems to correspond to Berlyne's concept of conflict. One proposed source of such disturbances is an "unfamiliar combination of familiar things (fear of the strange)" \cite[][, p. 250]{hebb_organization_2002}, which can disrupt normal responses to either of the familiar things, i.e., create conflict. Hebb's physiological theory, in turn, is in line with the Yu and Dayan's more recent and more precise theories of uncertainty processing in the brain \cite{yu_expected_2003,yu_uncertainty_2005}.

% Uncertainty - a need for mechanistic explanation of the origins of uncertainty
Theoretical and empirical studies above show us how quantitative variability in situational unpredictability is related to forward curiosity, but the precise mechanism(s) underlying this relationship remain speculative and await empirical validation. Additionally, while Yu and Dayan's account explains how uncertainty and incongruity can be computed, neurophysiological implementation of these computations is yet to be discovered. Furthermore, although the involvement of cholinergic and noradrenergic systems is likely, we still lack the description of precise neural pathways from uncertainty and surprise processing to subsequent motivational and affective responses that we experience as curiosity. These are promising avenues for future research on neural mechanisms of curiosity.

% A word on novelty here?


% What happens after uncertainty?
Having discussed the situational factors influencing curiosity we can now turn to how the subsequent state unfolds. We can identify two distinct aspects of that state. The affective-emotional aspect pertains to how curiosity is experienced from a subjective view-point. The motivational aspect concerns what the individual does in response to uncertainty.

\subsection{Affect and motivation} 

% "Aversive curiosity" -- too simple 
Curiosity, was long assumed to be the motivation to reduce aversive states. It was explicitly characterized as a drive that people are motivated to eliminate \cite{berlyne_theory_1954,loewenstein_psychology_1994}. Observations that uncertainty can induce fear and anxiety \cite{hebb_organization_2002,carleton_fear_2016} seemed to support this view. However, since perception itself is fundamentally an inferential process \cite{olshausen_perception_2013,friston_active_2016}, varying degrees of uncertainty underlie much of our subjective experience. Without making additional assumptions, uncertainty --  like any other attended state -- must undergo a cognitive appraisal that determines the affective profile of the ensuing state \cite{anderson_relationship_2019}. This implies two conclusions about the affective aspect of uncertainty and curiosity.

% Uncertainty is affectively neutral
First, uncertainty has to be appraised in order to induce affect. It may also fail to elicit an affective response if deemed irrelevant after the appraisal. There are many things about which we are knowingly ignorant; and even if we have some faint knowledge about such things, we do not necessarily judge our ignorance as positive or negative. I doubt that not knowing the first letter of my great-grandfather's name provokes any kind of emotion in the reader. Besides, if uncertainty was anxiety or fear-inducing, young children would have to be in a constant emotional distress. In order to elicit any affective response, the situation's personal relevance -- its significance for one's goals and needs -- must be evaluated \cite{lazarus_progress_1991,cunningham_motivational_2012}. For instance, accounts that portray uncertainty as anxiety-inducing assume the presence of perceived threats \cite{grupe_uncertainty_2013}. Without referring to peoples' current goals and needs it is simply impossible to make a general statement about how they feel about uncertainty.

% Curiosity can be positively or negatively valenced
Second, the possible motivational state resulting from uncertainty can be accompanied by positive and negative feelings alike. For instance, participants in Noordewier and van Dijk's study \cite{noordewier_curiosity_2017} reported positive feelings about curiously anticipating an intriguing video clip but only when they expected to watch it after only one minute of waiting, as opposed to those who expected a 30-minute delay and experienced their curiosity negatively \cite[but see][]{van_lieshout_curiosity_2020}. These discrepancies are in line with a more recent theory of I/D-type curiosity \cite{litman_curiosity_2019}, which maintains that curiosity can be experienced as a feeling of deprivation (D-type) or a feeling of interest (I-type). The D-type curiosity is associated with negative affect, while the I-type is experienced positively. This I/D-type taxonomy resonates with contemporary views on curiosity and interest \cite[e.g., ][]{hidi_interest_2019,shin_homo_2019,murayama_process_2019}, which propose that there are two distinct affective/motivational states that support information-seeking behavior. One state is accompanied by an unpleasant feeling of a knowledge-gap and another is characterized by a pleasant anticipation of learning. Although they are distinct, it is not very common for these states to coincide \cite{hidi_interest_2019}.

% Motivational salience
A useful framework for understanding the motivational states underlying non-instrumental information-seeking is the \textsc{incentive motivation} theory \cite{berridge_wanting_2009,robinson_roles_2016}. The central tenet of this theory is a distinction between systems "wanting" and "liking", also referred to as \emph{incentive salience} and \emph{hedonic impact}, respectively. Traditionally, the two terms are marked with quotes to distinguish them from more intuitive, everyday meanings of wanting and liking. Incentive salience refers to the visceral desire usually directed at a specific state or stimulus. \graffito{\emph{Intentionality} refers to the contents or representational targets of certain mental states, their "aboutness" or "directedness".} Notably, "wanting" is distinguished from cognitive desire -- a more explicit and \emph{intentional} kind of wanting that involves conscious thoughts about the object of desire and its emotional significance, akin to goal representation. On the other hand, hedonic impact refers to the objective hedonic responses to stimuli such as activation of hedonic "hotspots", certain facial expressions, and other physiological manifestations of subjective pleasure \cite{berridge_pleasure_2015}. As such, hedonic impact is distinguished from the subjective conscious feelings of liking. Normally, "wanting" coincides with cognitive desire and both are intricately related to "liking" and pleasant feelings that tend to also go hand in hand. While the unconscious "wanting" and "liking" systems might be phylogenetically and ontogenetically older, conscious desires and feelings confer additional evolutionary value by enabling more flexible and allostatic behavior \cite{damasio_nature_2013}.

% Distinct roles of wanting and liking
Incentive salience and hedonic impact are implemented by anatomically overlapping but functionally distinct systems in the brain \cite{berridge_dissecting_2009}. The "wanting" system resides in the circuitry that includes midbrain DA nuclei and their mesolimbic projections. It is responsible for maintaining seeking behaviors that can be directed towards rewards and reward-conditioned cues, but can also be completely non-intentional \cite{berridge_wanting_2009}. The "liking" system includes relatively small neuronal populations distributed throughout the frontal and insular cortices and the midbrain \cite{berridge_pleasure_2015}. It functions to communicate value of experienced states to other systems in the brain \cite{damasio_nature_2013}. These distinct functions, as one might expect, are highly complementary: hedonic responses signal what the organism should want. However, the activity of the two motivational components is dissociable, which can lead to confusing behaviors, such as "wanting" something without "liking" it or "liking" something without having "wanted" it. Since voluntary information-seeking is a type of motivated behavior, the incentive motivation theory has a potential to explain why people can be "seduced" by either pleasant or unpleasant information \cite{fitzgibbon_seductive_2020}.

% Relevant tenets of incentive motivation
Incentive motivation theory has gradually replaced the previously dominant drive theories of motivation on the grounds that drive reduction is a weak (negative) reinforcer at best  \cite{berridge_evolving_2018}. According to the incentive motivation theory, while the unpleasantness of a drive (e.g., a feeling of deprivation) can be experienced -- and to a high degree -- it does not spark or fuel motivation, but serves to intensify it. Recall that the affect associated with the "wanting" state is determined by an orthogonal appraisal process, as discussed earlier. All this implies that "wanting" information is not an inherently unpleasant feeling, as drive theories of curiosity \cite{berlyne_theory_1954,loewenstein_psychology_1994} have assumed. Information-seeking is positively incentivized by information and can be modulated by uncertainty that can be affectively positive, negative, or neutral. Crucially, the desired information does not have to be available to activate the "wanting" system, just like we don't need to see a cup of coffee in order to start craving one. As mentioned earlier, incentive salience can be initiated by learned associations between rewarding stimuli and arbitrary cues.

% Implications for curiosity and information-seeking
Uncertainty can be regarded as a cue that has a power to activate the "wanting" system, while information that resolves uncertainty can be viewed as input to "liking" system. As noted above, the degree of uncertainty can have a moderating effect on the intensity of "wanting". Several empirical findings support this view. In addition to the "observing task" study reviewed earlier \cite{bromberg-martin_midbrain_2009}, Aron et al. \cite{aron_human_2004} found a correlation between categorization uncertainty and midbrain activity. Gruber et al. \cite{gruber_states_2014} report a positive relationship between curiosity ratings about trivia questions and activation levels in the \ac{DA} midbrain and the \ac{NAc}. White et al. \cite{white_neural_2019} used single-cell recordings to show graded activity related to reward uncertainty in the \ac{DS} and the \ac{VP}, both involved in incentive salience \cite{smith_ventral_2009,volkow_nonhedonic_2002}. These are only indirect clues that raise the appeal of the idea that uncertainty moderates information-"wanting". Several other studies on human curiosity do not report associations between uncertainty/curiosity and the incentive salience system \cite{kang_wick_2009,jepma_neural_2012,van_lieshout_induction_2018}. To make things clearer, we need more research focusing on the relationships between degrees of uncertainty, curiosity, and incentive salience.

% Liking information is the foundation for motivated information-seeking responses
How does uncertainty get associated with information to become a cue for information-"wanting"? The answer to this question is likely to explain individual differences in personality traits like tolerance to uncertainty \cite{hillen_tolerance_2017}, need for cognition \cite{cacioppo_need_1982}, developed individual interests \cite{hidi_four-phase_2006} and more. The mechanism proposed by the incentive motivation theory is conditioning; we might as well call \ac{RL} \cite{maia_reinforcement_2009}. Actions that reduce a previously registered state of uncertainty are reinforced by "liked" states or stimuli and discouraged by "disliked" ones. The agent can thus develop abstract generalizable strategies (e.g., read a book, search in Google, ask a parent, etc.) to obtain information cued by uncertainty in various contexts. Interestingly, the theory of incentive motivation suggests that through this associative process, uncertainty itself can eventually become the target of "wanting" so that the agent will be willing to work to get to this state. However, it should be emphasized that it is the initial "liking" of information that is rewarding, and as such, it is the foundation of motivated information-seeking in response to uncertainty. An intriguing and empirically testable prediction follows: without sufficiently frequent pairing of uncertainty and "liked" information, no appreciation of uncertainty states can develop in a non-instrumental setting.

\section{"Liking" information}\label{sec:liking_information}

% Back to skill acquisition
Let us now take a step back and recall the idea that valuing non-instrumental information is biologically adaptive because it enables individuals to develop large and diverse sets of skills via knowledge acquisition. The motivational mechanism of uncertainty-triggered, non-instrumental information-seeking provides a plausible but not very detailed explanation for how knowledge acquisition might achieved in humans, and perhaps other animals. To render a more complete picture, we need to understand the sufficient conditions for "liking" information. This is by no means an unexplored territory, as researchers have proposed several potential explanations. It is important to mention that different propositions reviewed below are compatible and should not be regarded as alternative general explanations of motivated information-seeking. A more productive mindset is to consider how different aspects of informational appeal determine situational curiosity and the development of more persistent interests \cite{kobayashi_diverse_2019,bromberg-martin_value_2020,fitzgibbon_lure_2021}.

% The 'savoring' account and related evidence
One proposition is that information is "liked" because it induces an inherently pleasing anticipation about a desired event in the future -- a phenomenon known as \emph{savoring} \cite{loewenstein_anticipation_1987}. The idea is that information that promises extrinsically valuable stimulation initiates a pleasurable anticipatory response that. This potentially explains why monkeys \cite{bromberg-martin_midbrain_2009,bromberg-martin_lateral_2011} and humans \cite{van_lieshout_induction_2018,kobayashi_diverse_2019} actively sample cues that might reveal the delivery of reward in the future. The "savoring" account has been empirically validated in humans and other animals. For instance, Spetch et al. \cite{spetch_suboptimal_1990} and Gipson et al. \cite{gipson_preference_2009} demonstrated that pigeons prefer to savor (and subsequently receive) an uncertain reward over receiving a more certain \cite[up to 100\% certain in][]{spetch_suboptimal_1990} reward over the same time delay, which shows their willingness to sacrifice extrinsic reward for information. This finding was later replicated in humans \cite{iigaya_modulation_2016}, albeit with the same 50\% reward probabilities in savoring and uncertain-waiting options. Iigaya et al. \cite{iigaya_modulation_2016} introduced a model that explain increasing preferences for advance information with longer waiting time. Notably, their computational model assumes that information-seeking also depends on \acp{RPE}, but only as a moderator (or "booster") of savoring. In this view, the total value of information depends on both extrinsic and intrinsic value.

% Savoring as a subclass of hedonic value of information
The "savoring" account of "liking" information can be viewed as an instance of a broader phenomenon: information can be "liked" for its pleasing implications. For example, some pieces of information support desirable beliefs that people are personally invested in \cite[e.g., confirmation bias][]{nickerson_confirmation_1998} -- let us call it the "hedonic value" conjecture. This taste for self-pleasing information might have adaptive value beyond knowledge-acquisition, for example, by elevating self-efficacy \cite{bandura_self-efficacy_1977} (the subjective belief if achieving certain outcomes in certain situations) for accomplishing desirable tasks \cite{bromberg-martin_value_2020}.\graffito{Note:, no single "conjecture" can account for all of motivated information-seeking.} This idea is certainly plausible but the "hedonic value" conjecture is insufficient for explaining many instances of non-instrumental information-seeking where information does not inherently imply anything positive for the individual. One example is counter-factual information: humans \cite{fitzgibbon_lure_2021} and monkeys \cite{wang_monkeys_2019} sample information about what happened or could have happened in the past -- something that can actually induce negative feelings of regret. Other examples include explanation-seeking \cite{coenen_asking_2019,liquin_explanation-seeking_2020} and exploratory play \cite{cook_where_2011,chu_play_2020}. In all of these cases, individuals seem to seek observations that convey accurate rather than self-serving information about the world. Intuitively, "liking" such information should facilitate autonomous knowledge acquisition for the development and diversification of skills, as I discussed in the first section of this chapter.

% Knowledge in computational literature
To advance the idea that "liking" of information that improves knowledge, we need to characterize "knowledge" more precisely. In cognitive-computational literature, knowledge emerges from parameterized (i.e., variable) models that optimize responses to stimuli \cite[e.g., connectionist and probabilistic cognitive models models][]{mcclelland_letting_2010,griffiths_probabilistic_2010}. Regardless of algorithmic and implementations details, the function (or outcome) of knowledge in computational systems is the same: enabling (probabilistic) inferences under uncertainty. These inferences can be perceptual (what is "out there"?), causal (what has lead to the current state?), predictive (what will be the next state?), or instrumental (what to do to get to desired state?). Following this more precise characterization of the construct of knowledge, we can now introduce the "epistemic value" conjecture -- information is "liked" when improves inferences.

% Knowledge improvement
Note that inference improvements must be subjectively represented for one to appreciate them. Inference $A$ can be better than inference $B$, because, for example, $A$ predicts the future more accurately than $B$. However, it is the subjective comparison that influences one's "liking" response. The outcome of subjective evaluation of inferences is uncertainty. Thus, information is "liked" when it subjectively reduces uncertainty. It is worth reiterating that in this view, uncertainty is a subjectively appraised state of ignorance that modulates the "wanting" of information (not an inherently negative drive), while the state representing uncertainty-reduction elicits a "liking" response. Thus, uncertainty can be gradually associated with positive feelings (e.g., interest) if it is consistently paired with knowledge improvement (and, perhaps other positive outcomes). The association can even become strong enough to make the state of uncertainty itself "wanted", though not necessarily "liked".

\section{Learning progress}

% Learning progress in AI
The "epistemic value" conjecture is closely related to the concept of \ac{LP} originating in \ac{AI} \cite[e.g., ][]{schmidhuber_curious_1991,oudeyer_intrinsic_2007}. \ac{LP}-inspired intrinsic-reward functions push agents to progressively explore their environments by focusing their resources on learning situations (see \autoref{ch:ai}) where they think their knowledge can be improved. Thus, despite revolving around the principle of prediction-error reduction, \ac{LP}-based reward functions are similar to other so-called heterostatic adaptive reward functions, like competence-progress or information-gain motivations \cite{oudeyer_what_2007}. At a higher level, all of these algorithms reward agents for reaching states where their knowledge is improving. Moreover,  algorithms have a common operation of estimating an expectation for some measure of the system's knowledge. 

% Self-efficacy
Expectations about self-improvement are also related to self-efficacy \cite{bandura_self-efficacy_1977}. Evidence for improving performance  -- a form of \ac{LP} called \emph{competence progress} \cite[see ][]{oudeyer_what_2007} -- might foster feelings of self-efficacy which bear on the psychological need for competence \cite{blain_intrinsic_2021} postulated by the Self-Determination Theory \cite{ryan_self-determination_2000}. According to this view, mastery-oriented effort is fueled by self-efficacy beliefs that are, in turn, supported by \ac{LP}. This self-efficacy account of intrinsic motivation adds important details to the current theoretical perspective. Not only it provides an explanation for why information that results in \ac{LP} is "liked" and "wanted" at the ontogenetic level, but also links this explanation with a psychological-need construct postulated on the basis of the theory of evolution \cite{ryan_self-determination_2017}. 

% Computational challenges
By re-engaging in situations where their predictions or competence are improving, artificial agents are able to efficiently address two non-trivial computational challenges of open-ended learning \cite{gottlieb_information-seeking_2013}. The first challenge is the virtually infinite amount of things one can learn and the concurrent limited resources of individual organisms. Given an ability to perceive similarity of various situations, \ac{LP}-guided agents can fragment the vast learning space and locate "progress niches" in it that are appropriate for their level of knowledge/ability \cite[e.g. ][]{oudeyer_intrinsic_2007,forestier_intrinsically_2020,etcheverry_hierarchically_2021}. The second challenge is the constant prevalence of uncorrelated events and unreachable goals that agents cannot learn or learn to achieve in principle. The strategy of pursuing \ac{LP} helps agents divert their resources from trying to improve predictability of uncorrelated variables and attempting to reach states that agents are not ready to reach or will never be able to reach \cite[e.g., ][]{forestier_intrinsically_2020, colas_curious_2019}. Furthermore, progress-based heuristics for time allocation across multiple tasks have been shown to be optimal in certain conditions \cite{son_metacognitive_2006,lopes_strategic_2012}. Thus, \ac{LP}-based motivation presents a potential solution to computational challenges faced by artificial and biological agents alike \cite{gottlieb_information-seeking_2013,gottlieb_towards_2018,oudeyer_computational_2018}.

% LP models are good for generating hypotheses
\ac{LP}-based reward functions from \ac{AI} provide testable algorithmic descriptions for computing subjective knowledge improvement. Being formally expressed, these descriptions render the underlying assumptions and representational requirements more transparent and unequivocal. These characteristics make algorithmic models of \ac{LP}-based motivation good sources for generating testable hypotheses. Despite implementational idiosyncrasies of different \ac{LP} algorithms \cite[see][]{oudeyer_intrinsic_2007,linke_adapting_2020}, it is possible to derive a common behavioral prediction. If people "like" information that results in \ac{LP}, they should be motivated to engage in situations where they think \ac{LP} will occur. This implies that stimuli of intermediate complexity" and activities of intermediate difficulty should engage people more compared to stimuli/activities of extreme complexity/difficulty. Few empirical studies focused on this particular "\ac{LP} hypothesis", but several behavioral results are compatible with this idea.

% Instrumental information-seeking
Before proceeding to the more pertinent work on non-instrumental information-seeking (e.g., curiosity studies), let me first discuss how the \ac{LP}-hypothesis relates to exploration in instrumental settings. It is worth examining information-seeking in instrumental contexts because it might have similar mechanisms with non-instrumental information-seeking \cite{gottlieb_towards_2018,gottlieb_curiosity_2020}. In the so-called "multi-armed bandit" task \cite[see ][]{averbeck_theory_2015}, participants sample from 2 or more sources of stochastic rewards, called "bandits". Of interest is how participants use their finite sampling opportunities to maximize the amount of total reward. On any given sampling trial, they can sample from bandits that they believe are most rewarding (i.e., exploit) or try less familiar ones in order to learn more about them (i.e., explore). Several studies report that often it is common for people to explore bandits that they are least certain about \cite{speekenbrink_uncertainty_2015,gershman_deconstructing_2018,schulz_structured_2019}. Importantly, uncertainty pursued by participants is the posterior uncertainty regarding expected-value estimates of the bandits and it is distinguished from the irreducible kind of uncertainty that we know as expected uncertainty \cite{schulz_algorithmic_2019}. Typically, prior expectations about the expected values are assumed to be at most weakly informative, which implies that the attractive posterior uncertainty of unfamiliar bandits is reducible. A bandit of maximal (reducible) uncertainty offers the most \ac{LP}, because low-entropy priors (more strongly held beliefs) are less likely to be influenced by new data compared to less informative priors. Thus, uncertainty maximization strategies observed in instrumental exploration contexts of multi-armed bandits seems compatible with the \ac{LP} hypothesis. This interpretation of the bandit literature can be tested in a study that matches the posterior uncertainty of one bandit and the expected uncertainty (risk) of another, while keeping the entropy of the risky bandit's posterior low. Finding that choices are random or trending towards the risky bandit would indicate that people are attracted by reducible rather than irreducible uncertainty, which would be evidence against the \ac{LP} hypothesis, at least in an instrumental setting. 

% Lottery experiments
Several results from research on non-instrumental information-seeking are also compatible with the \ac{LP} hypothesis. For instance, studies by van Lieshout et al. \cite{van_lieshout_induction_2018} and Kobayashi et al. \cite{kobayashi_diverse_2019} report monotonic relationships between uncertainty and motivation for information-seeking. Both studies gave participants risky lotteries which could win them some amount of money. Participants could sample accurate information about the amount of the lottery in advance by paying some known cost. Since the experimenters delivered the promised information reliably upon request, participants could expect to minimize their prediction about the upcoming reward (without actually influencing the reward amount or delivery time). Therefore, in these tasks, maximal-uncertainty lotteries can be reasonably speculated to correspond to learning situations with highest expected \ac{LP}. These results are also compatible with intrinsic-motivation algorithms where agents are driven to maximize and seek uncertainty \cite[see ][]{oudeyer_intrinsic_2007,linke_adapting_2020}. However, these algorithms have known deficiencies in complex stochastic environments featuring sources of irreducible uncertainty. 

% Trivia questions
Studies using trivia-question paradigms seem to be at odds with the lottery tasks reviewed above. Specifically, Kang et al. \cite{kang_wick_2009} and Baranes et al. \cite{baranes_eye_2015} reported an inverted-U relationship between uncertainty in knowing the answer to a question and self-reported curiosity and information-sampling measures -- participants were more interested in questions for which they reported intermediate certainty of knowing the answer. This finding fits the \ac{LP} hypothesis better as it predicts the strongest motivation for intermediate-complexity stimuli. The apparent contradiction with van Lieshout et al. \cite{van_lieshout_induction_2018} and Kobayashi et al. \cite{kobayashi_diverse_2019} can be explained by noting the crucial differences in experimental stimuli. In contrast to the trivia-question task, learning situations that each trial of the lottery task presents are unrelated to participants' knowledge beyond the experiment. Though it might seem that answers to trivia questions can be memorized in one shot, committing declarative propositions to one's memory (i.e., learning them) depends on prior knowledge \cite{brod_influence_2013}. Intermediate-confidence questions could be the most interesting because they are more likely to be committed to memory, given one's existing knowledge. In fact, things that are uncertain yet associated with one's prior knowledge -- to the point that one can have a ballpark idea of the correct answer -- are more interesting and learned better \cite{brod_lighting_2019}. While findings from both trivia-question and lottery tasks are compatible with the \ac{LP} hypothesis, they are not designed to probe knowledge acquisition over extended periods of time, as it happens in natural settings.

% Time extended learning paradigms
Several studies of active, time-extended information-seeking exist. Gerken et al. \cite{gerken_infants_2011} showed that infants attend to stimulus sets with learnable structure longer compared to sets with unlearnable structure. Kidd et al. \cite{kidd_goldilocks_2012} showed that infants look longer at sequences of intermediate-complexity compared to highly predictable and excessively unpredictable sequences. In a similar study, Poli et al. \cite{poli_infants_2020} provide a more direct support to the \ac{LP}-based engagement by showing that infants' tend to look away from stochastic sequences when they no longer update an assumed predictive model. Geana et al. \cite{geana_boredom_2016} showed that human adults rate invariant and random-uniform number generators as more boring than random-but-predictable number generators based on normal distributions. Leonard et al. \cite{leonard_young_2021} showed that young children who saw evidence for gradual improvement were more likely to stick to a challenging task compared to children who's improvement record was held constant.  While findings from these studies are compatible with and even more relevant for the \ac{LP} hypothesis, they all measure task engagement by first forcing learning situations on participants and then measuring the likelihood of task disengagement. On the other hand, the \ac{LP} hypothesis predicts how people decide to actively engage in tasks themselves.

% Metacognition research
A few studies have examined the relationship between free active engagement in time-extended learning tasks and variables closely linked to \ac{LP}. Son and Metcalfe \cite{son_metacognitive_2000} gave participants multiple biographical texts to explore over a fixed amount of time. Before free exploration, participants sampled each text and provided metacognitive judgments for how easy they thought the texts were. Participants spent more time on texts that they judged to be easier, and also sampled these texts earlier.\graffito{In metacognition research, a \ac{JOL} is a self-reported confidence in retrieving a stimulus upon request in the future.} Metcalfe and Kornell \cite{metcalfe_region_2005} subsequently reported a positive correlation between study time and a crude temporal derivative of explicit \acp{JOL}: people spent more time studying Spanish-English word pairs for which their judgments of learning were increasing. Interestingly, Metcalfe and Kornell's \emph{Region of Proximal Learning} theory \cite{metcalfe_region_2005} predicts learning activities are given priority with respect to their \ac{JOL}, similarly to how uncertainty can cue and modulate the motivation for information-seeking. Their theory also predicts that perseverance on a learning activity is determined by a temporal derivative of \ac{JOL}, which resonates with the idea that low-\ac{LP} do not arouse interest. While Metcalfe and colleagues' work is clearly related to the \ac{LP}-hypothesis, task designs in the reviewed studies did not permit a close tracking of learning rates and task engagement.

% Call for research
It is still unclear whether human brains feature mechanisms to compute \ac{LP} and generate motivation for seeking it. Considering the necessary computational features of \ac{LP} algorithms as well as a number of studies that are compatible with their operation, studying the role of \ac{LP} in a dedicated task is an important step in advancing our understanding of information-seeking.

% "Monster" study
In \autoref{ch:learning-progress}, I report an original study that introduces a behavioral paradigm for studying concurrently unfolding learning and motivated exploration of multiple non-instrumental activities. Using a computational model of trial-by-trial choice utility, we demonstrate that humans do show sensitivity to \ac{LP} while exploring multiple learning activities. I also discuss limitations of this early attempt to measure \ac{LP} and its relation to self-guided choice of activity. These limitations are addressable by minor changes to the parameters of our paradigm and highlight the need and promising paths for future research.

% "Lunar Lander" study
One of the discussed limitations is our poor understanding of metacognitive mechanisms for generating subjective judgments of improvement. In \autoref{ch:metacognition}, I review some of the related work in metacognition research, identify several candidate computational mechanisms, and propose a novel behavioral paradigm for collecting data to help us evaluate these mechanisms empirically. I also present some exploratory data from a pilot study using this paradigm and discuss directions for future improvement.

% Next chapter
Before proceeding to the original empirical contributions of this thesis, I will review the state-of-the-art in computational literature on intrinsically motivated learning. The importance of non-instrumental exploration has been increasingly recognized as a key component of open-ended autonomous learning in artificial systems. While this recognition is inspired by how biological agents seem to learn, the rapidly evolving field of Machine Learning has been successful at proposing and testing numerous original computational mechanisms for active learning in the absence of traditional problem-specific reward functions. Although there is considerable thematic overlap with psychological theories, computational literature can be distinguished by its adherence to (1) formal specification of computational problems facing agents embedded in specific environments and (2) simulations of agent-environment interactions under the assumed mechanisms and environmental contexts. This approach not only constrains the search for potential mechanisms, but also helps to understand the functional significance of specific mechanisms in specific problem-settings \cite[e.g., ][]{lopes_strategic_2012,moulin-frier_exploration_2013}. Clear presentation of core computational challenges of information-seeking in humans can help us make sense of multiple empirically supported propositions about the underlying mechanisms \cite{dubey_reconciling_2020,brandle_what_2020} and direct our inquiry towards the right questions \cite{coenen_asking_2019}.