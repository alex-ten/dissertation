%************************************************
\chapter{Psychology}\label{ch:psychology}
%************************************************

% This section should discuss different aspects of curiosity that psychologists study
\section{Many faces of curiosity}

"Curiosity" in everyday language can mean slightly different things: We are curious about an event => we want to know;  When we ask something, "just out of curiosity" => wanting to know, not because of knowing is useful for someting, but because it knowing itself is rewarding; We are a curious person => generally inquisitive (like a cat); We can also attribute curiosity to events or even things: a curious case of Benjamin button; A curious substance/device. 

Psychologists (including computational and neuroscience people) study different aspects of curiosity: states and traits. Psychological traits characterize our behavioral tendencies. When someone is "open to experience" they are more likely to try out a new genre of music; these tendencies describe how an individual is similar or different to others. States on the other hand are particular configurations of a physical system. Two individuals might have different personalities, they are expected to behave similarly if they are both in the same state. The body of research on different aspects of curiosity is huge, and it helps to be aware of what exactly a given piece of research investigates. While research on trait curiosity is rich and important, I'll focus on the state that people refer to as curious or curiosity.

The state of being curious has multiple levels at which it can be described. Phenomenologically, it is one thing. Neuronal state, the state of the brain at the level of neurons is another. A psychological description of the neuronal state is what allows us to link what we can observe objectively (brain recordings) and what we feel subjectively (your first-hand experience). It is an interpretation of neuronal data, which does not make much sense without invoking psychological constructs like uncertainty, predictions, incongruence, etc. In general, when I say "state" I refer to all three of these descriptions. Note that some states are unconscious in a sense that it does not feel like anything to be experiencing these states. Phenomenologically, these states are empty, but they are still observable and interpretable objectively.

% This section will serve as a meta-review of existing review articles. A normal descriptive or even critical review tries to integrate findings from different researchers into a coherent picture. The reviewers then pinpoint which questions (unknowns/problems) are important to address, what study designs are potentially useful to address them, what problems the field faces, what are the correct approaches to studying something etc. One could say that such reviews contribute to the formation of a "mini-paradigm": a set of problems (research questions) and corresponding solutions (hypotheses and theories) that are concensually exemplar. A review tries propose such problems and solutions to the readership in an attempt to create some concensus. 
\section{Current views}
\citeauthor[]{murayamaProcessAccountCuriosity2019,bazhydaiCuriosityExploration2020,gottliebCuriosityInformationDemand2020,cerveraCuriosityPerspectiveSystems2020,renningerCambridgeHandbookMotivation2019,kiddPsychologyNeuroscienceCuriosity2015,gottliebNeuroscienceActiveSampling2018,silviaLookingCuriousPersonality2020,kashdanCuriosityInterestBenefits2009,metcalfeEpistemicCuriosityRegion2020} + berlyneATheoryOf1954

Berlyne does not explain why some questions might elicit more or less curiosity. Loewenstein seems to explain this by formulating curiosity as a "reference-point phenomenon". Loewenstein, however, does not explain how reference points are chosen (or does he, read again)

\section{Learning Progress}
Learning progress is implicit in many theories of curiosity. Loewenstein's information gap theory, for instance, seems to predict that curiosity (the desire to know) scales with the size of the perceived information gap. This can be seen as a central prediction of this theory. However, Loewenstein also makes another prediction: that curiosity-induced information-seeking response-probabilities (e.g., requesting answers to n out of N questions) will depend on the ability of these response to close the gap (i.e., the learning progress that a response or behavior promises).