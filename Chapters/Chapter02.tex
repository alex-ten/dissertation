%************************************************
\chapter{Psychology}\label{ch:psychology}
%************************************************

\begin{flushright}{\slshape
    Where is the wisdom we have lost in knowledge? \\
    Where is the knowledge we have lost in information? \\
    --- T. S. Eliot ("The Rock", 1934)}
\end{flushright}
The above quote by T.S. Eliot is one of the earliest sources that inspire the famous Data-Information-Knowledge-Wisdom model (DIKW; Sharma, 2008) -- a conceptual framework that draws distinctions and defines relationships between data, information, knowledge, and wisdom. While the DIKW model is tossed around in the context of business management and decision-making, distinguishing between information, knowledge, and wisdom is also important in the scientific study of information-seeking. This chapter reviews the current literature (mostly) about non-instrumental information-seeking and presents a theoretical perspective on what makes information inherently valuable for individuals from a curious species like humans.

\section{Information and information-seeking}

Sense organs are ubiquitous in the living world. Indeed, they are so useful that not only animals but plants, fungi, and multiple other organisms have evolved them \cite{trewavas_plant_2005,braunsdorf_fungal_2016,bourret_census_200,schwab_evolution_2018}). Moreover, humans endow their technology with artificial sensors to make "smart" devices. %[a picture with sensors would be nice]. 
It is no mystery why sensors are so powerful: they \emph{inform} their owners about what is "out there" so that they can adapt their behavior accordingly. Information, then, is the basis of intelligent behavior. It is also a central concept in this thesis, so we need to establish what it means more precisely. 

Information is an important concept across many disciplines. Even though precise definitions differ, it is possible to identify three major conceptual stances on what information means \cite{adriaans_introduction_2008}. According to one perspective, information refers to declarative descriptions of the mentally represented world, which can be obtained, for example, through empirical observation, linguistic communication, or "armchair" deduction. This is the sense in which the word 'information' tends to be used in lay discourse. Another view characterizes information in terms of uncertainty. Here, information is viewed as an abstract communication process by which uncertainty about some random event can be reduced. This formulation is commonly adopted in Information Theory \cite{shannon_mathematical_1948}. Finally, there is an approach that treats information as the complexity of the simplest possible representation of an object in a given (i.e., fixed) system. This is a Kolmogorov-complexity stance on information \cite{kolmogorov1965three}. It underlies the intuition that simple objects require less information to be described in, say, natural language or neural code, compared to complex objects.

While these three perspectives may seem quite far apart, they are demonstrably and rather intricately connected \cite{adriaans_introduction_2008}. For example, sensing can be viewed as a process that reduces an organism's uncertainty (stance 2) about the environment by representing and possibly compressing it (stance 3) and thus \emph{forming} an \emph{in}ternal description of the surrounding world (stance 1). In other words, sensing is a process of communicating, representing, and interpreting information\footnote{Information goes beyond sensing. We can talk about paintings, texts, or even brains "containing" information (stance 1). Mnemonic retrieval of information can be viewed as a kind of "internal sensing" by which one system "observes" and encodes infromation from another, all within the same brain.}

Since animals are capable of moving around, they can reposition and orient their sensors to different parts of the world. Additionally, because some animals can also exert considerable forces on their surroundings, they are also capable of acting \emph{on} the world to expose their sensors to specific stimulation. This \emph{active} sensing behavior can be characterized as information-seeking. 

Notice how broad this characterization of information-seeking is. It implies that behaviors that do not normally strike us as information-seeking can be regarded as such. For instance, following the scent gradient of food is an instance of information-seeking because it involves interactions between acting and sensing. It seems like sensory information is needed for almost everything we do, yet, since the world is stupenduously complex, there is probably more potentially observable or contemplatable information in it then organisms can possibly represent and process \cite{kolmogorov1965three}. 

Given the costs and constraints of movement and computation in biological organisms, the overabundance of potential information in the world means that it needs to be somehow funneled \cite{gottlieb_towards_2018} for senses to be useful. Physical features of sense organs responsible for domain specialization (e.g., light, sound, pressure) and sensitivity to specific intensity ranges within domains (e.g., \cite{schwab_evolution_2018}) can be viewed as passive, structural funnels of information. Another, active kind of funneling is enabled by the ability to selectively expose one's external and internal sensors to specific stimulation via actions and neuromodulation. Much of what organisms decide to attent do is not random but depends on several situational factors including environmental stimuli, individual knowledge, emotions, and transient goals\footnote{Goals and emotions might in turn be rooted deep in stable biopsychological needs \cite{ryan2017self}}. Understanding how these factors interact (neurally and psychologically) and how their interaction determines attention is the overarching objective of scientific research on information-seeking.

A well-established approach to organizing information-seeking behaviors is to classify them into two broad categories \cite{gottlieb_information-seeking_2013,gottlieb_towards_2018}. \emph{Instrumental} information-seeking includes behaviors driven by goals that cannot be achieved by sensory (or neural) stimulation alone. For example, although foraging for food (or tracking a predator) requires seeking information and active attention, the ulterior end is the consumption of nutrients (or avoidance of predation) -- something that cannot be achieved solely by observing or thinking thinging about food (or predators). \emph{Non-instrumental} information-seeking includes everything else. While it is possible to identify ends separable from information-consumption \emph{per se} (e.g., knowledge-gains, feelings of competence), information is sufficient for satisfying these ends\footnote{Note that while "instrumental/non-instrumental" and "intrinsically/extrinsically motivated" terminology is a convenient way to name classes of information-seeking behaviors, it is not fully consistent with the actual criterion for classificaiton. Indeed, the so-called non-instrumental behaviors and intrinsically motivated behaviors are at least assumed to be driven by outcomes separable from the consumption of information itself -- identifying such outcomes is what many researchers actually pursue. I will keep using this terminology in this thesis, but the reader should keep in mind what the real point of distinction is.}. While it is fairly easy to identify external goals in instrumental contexts, understanding what drives non-instrumental information-seeking is less obvious.

% Only outline of paragraphs from here on 

\section{Non-instrumental value of information}

    Disucss different aspects of what needs to be studied: 
    
    1) Neural mechanisms of attentional control (how does the brain know where to look? -- Gottlieb and co). Highlight the concept of the "signal for control" (Gottlieb et al., 2020; Should I stay or should I go paper).

    2) Explain my research focus: what signals the need for non-instrumental information-seeking. (i.e. when there are not extrinsic rewards)\

    Introduce the subsections

    \subsection{Information as savoring}

        Hedonic value: Iigaya et al., (2018 -- model) 
            
        Observing task (Bromberg-Martin & Hikosaka, 2009, 2011 -- monkeys; Blanchard et al. -- value encoding), Kobayashi et al. (2019 -- humans)

        This is not everything -- we seek other kinds of information. Maybe discuss IIgaya et al. (2016 -- eLife) and Iigaya et al. (2019 -- BioRxiv), but close with Cervera et al., (2020) who argue that information has inherent value unrelated to rewards

    \subsection{Information as a relief}

        Berlyne, Loewenstein, Jirout, Hidi & Renninger, van Lieshout et al., propose that curiosity is aversive. Curiosity motivation hinges on the desire to minimize negative experience.

        Find a fMRI/cogsci paper where negative valence of uncertainty is demonstrated.

        Evidence: observing paradigm; van Lieashout (2018) and Kobayashi (2019)

        The problem of voluntary exposure to curiosity -- why interest exists if curiosity is aversive? Maybe, discuss coin flips. Maybe noisy TV problem. Relationship between curiosity and interest (Hidi & Renninger; Litman, maybe self-serving bias papers?)

    \subsection{Information as knowledge-gain}

        Theory -- knowledge-gain as intrinsic value of information (Murayama, 2019; Hidi & Renninger 2019)

        Foreshadow intrinsic motivation in AI chapter

        Learning progress and evidence (Leonard et al., Poli et al., Gerken et al., kidd goldilocks, kang + Baranes).

        Learning progress is implicit in many theories of curiosity. Loewenstein's information gap theory, for instance, seems to predict that curiosity (the desire to know) scales with the size of the perceived information gap. This can be seen as a central prediction of this theory. However, Loewenstein also makes another prediction: that curiosity-induced information-seeking response-probabilities (e.g., requesting answers to n out of N questions) will depend on the ability of these response to close the gap (i.e., the learning progress that a response or behavior promises).
