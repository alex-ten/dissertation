%************************************************
\chapter{Psychology}\label{ch:psychology}
%************************************************

"Where is the wisdom we have lost in knowledge? Where is the knowledge we have lost in information? 

The above quote from T.S. Eliot's play "The Rock" is one of the earliest sources that inspire the Data-Information-Knowledge-Wisdom model (DIKW, Sharma, 2008) -- a conceptual framework that draws distinctions and defines relationships between data, information, knowledge, and wisdom. While the DIKW model is tossed around in the context of business management and decision-making, the distinction between information and knowledge is also quite useful in the scientific study of information-seeking. This chapter reviews the current literature (mostly) about non-instrumental information-seeking and presents a theoretical perspective on what makes information inherently valuable for individuals from a "curious" species like humans.

\section{Information and information-seeking}

Sense organs are ubiquitous in the living world. Indeed, they are so useful that not only animals but plants, fungi, and micro-organisms have evolved them \cite{trewavas_plant_2005,braunsdorf_fungal_2016,bourret_census_2006}). Moreover, humans endow their technology with artificial sensors to make smarter devices. %[a picture with sensors would be nice]. 
It is no mystery why sensors are so powerful: they \emph{inform} their owners about what is "out there" so that they can adapt their behavior accordiingly. Information, then, is the basis of intelligent behavior. It is also a central concept in this thesis, so we need to establish what it means more precisely. 

Information is an important concept across many disciplines. Even though precise definitions differ, it is possible to identify three major conceptual stances on what information means \cite{adriaans_introduction_2008}. According to one view, information refers to declarative descriptions of the mentally represented world, which can be obtained, for example, through empirical observation, linguistic communication, or "armchair" deduction. This is the sense in which the word 'information' tends to be used in lay discourse. Another view characterizes information in terms of uncertainty. Here, information is viewed as an abstract communication process by which uncertainty about some random event can be reduced. This formulation is commonly adopted in Information Theory \cite{shannon_mathematical_1948}. Finally, there is an approach that treats information as the complexity of the simplest possible representation of an object in a given (i.e., fixed) system. This is a Kolmogorov-complexity stance on information \cite{kolmogorov1965three}. It underlies the intuition that simple objects require less information to be described in, say, natural language or neural code, compared to complex objects.

While these three stances may seem quite far apart, they are demonstrably and rather intricately connected \cite{adriaans_introduction_2008}. For example, sensing can be viewed as a process that reduces an organism's uncertainty (stance 2) about the environment by representing it (stance 3) and thus (in)\emph{forming} a description of the surrounding world (stance 1). In other words, sensing is a process of communicating, representing, and interpreting information. Since animals are capable of moving around, they can reposition and orient their sensors to different parts of the world. Additionally, because some animals can also exert considerable forces on their surroundings, they are also capable of acting \emph{on} the world to expose their sensors to specific stimulation. This \emph{active} sensing behavior can be characterized as information-seeking. 

Note, that this characterization of information-seeking is extremely broad. It implies that behaviors that do not normally strike us as information-seeking can be regarded as such. For instance, following the scent gradient of food or tracking the proximity of a predictor to keep away are both instances of information-seeking because they involve interactions between acting and sensing. Because active sensing is so ubiquitous, it is useful to group information-seeking behaviors into categories. A particularly topical organization is a two-branch taxonomy that classifies any given instance of information-seeking by referring the motivational factors responsible for its initiation. \emph{Instrumental} information-seeking includes active sensing initiated by an outcome separable from the consumption of information itself. For example, when searching for food or avoiding a predator, information about the food or the predator is a means to an end of consuming food or avoiding predation. On the other hand, \emph{non-instrumental} information-seeking includes active sensing initiated by the consuption of information as a terminal end. 

\subsection{Instrumental value of information}

    Examples of value of information

    Paradigms

    Relevant results (Mostly gottlieb -- see Recent review in Current opinion + Gottlieb et al., 2013 for references)

    Conclusions

    Mechanisms

\subsection{Non-instrumental value of information}

    Examples

    \subsubsection{Information as a relief}

        Try to find a paper where negative valence of uncertainty is demonstrated.

        Berlyne and other accounts proposing that curiosity is aversive. Curiosity motivation hinges on the desire to minimize negative experience.

        Evidence: observing paradigm; van Lieashout (2018) and Kobayashi (2019)

        The problem of voluntary exposure to curiosity -- why interest exists if curiosity is aversive? Maybe, discuss coin flips. Maybe noisy TV problem. Relationship between curiosity and interest (Hidi & Renninger; Litman, maybe self-serving bias papers?)

    \subsubsection{Information as savoring}

        Hedonic value: Iigaya et al., (2018 -- model) 
            
            Observing task (Bromberg-Martin & Hikosaka, 2009, 2011 -- monkeys; Blanchard et al. -- value encoding), Kobayashi et al. (2019 -- humans)

            This is not everything -- we seek other kinds of information. Maybe discuss IIgaya et al. (2016 -- eLife) and Iigaya et al. (2019 -- BioRxiv), but close with Cervera et al., (2020) who argue that information has inherent value unrelated to rewards

    \subsubsection{Information as knowledge-gain}

        Theory -- knowledge-gain as intrinsic value of information (Murayama, 2019; Hidi & Renninger 2019)

            Foreshadow intrinsic motivation in AI chapter

            Learning progress and evidence (Leonard et al., Poli et al., Gerken et al., kidd goldilocks, kang + Baranes).

            Learning progress is implicit in many theories of curiosity. Loewenstein's information gap theory, for instance, seems to predict that curiosity (the desire to know) scales with the size of the perceived information gap. This can be seen as a central prediction of this theory. However, Loewenstein also makes another prediction: that curiosity-induced information-seeking response-probabilities (e.g., requesting answers to n out of N questions) will depend on the ability of these response to close the gap (i.e., the learning progress that a response or behavior promises).
