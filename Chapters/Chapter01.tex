Why a particular behavior has evolved and why we do it can be different questions with different answers. There is a "big why" question that asked how did we as a species come to behave a certain way. Then there is a "small why question" that asks what makes a particular individual behave that way. The most likely but somewhat vague answer to almost any "big why" question about a behavior is selection. The behavior in question must have provided some kind of an advantage to its carriers so that it got to be passed on and thus propagated. How this happens may vary -- for some behaviors are passed on culturally -- but if a behavior is universal and present in the young, it might be safe to assume that it is somehow predetermined by our genes. The "small why" of why we do something is different. But like the "big why" question, it has a generally truthful but somewhat vague answer -- emotions. We tend to do things that bring us pleasure and avoid things that are unpleasant.

There are two loss functions that are being optimized in the process of shaping behavior. The biological fitness function is optimized by the selection of affective "reward loss functions" which individual organisms work to optimize during their lifetimes. The biological fitness function can select for entire trajectories of reward functions during a single lifetime, so that babies and adults act differently.