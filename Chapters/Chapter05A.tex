\begin{subappendices}

\chapter*{Appendix to Chapter 5}\label{CH5A}

Below are the questionnaire items used in the pilot study.

\section{Situated Intrinsic Motivation Scale (SIMS)} Responses were recorded on a 7-point Likert scale (1="Not at all", 2="Very little", 3="A little", 4="Moderately", 5="Enough", 6="A lot", 7="Exactly") indicating the extent to which the respondent agrees with the reason for engaging in the learning activity. Specifically, the prompt for every item read: "Read each item carefully. Using the scale below, please indicate \textit{how much each item describes the reason why you are engaged in this activity} (i.e., Lunar Lander game)".
\begin{itemize}
    \item Amotivation
    \begin{compactitem}
    	\item I do this activity but I am not sure if it is worth my time
    	\item I don’t know; I don’t see what this game brings me
    	\item There may be good reasons for practicing this game, but personally I don’t see any
    	\item I keep practicing, but I am not sure I should continue
    \end{compactitem}
    \item External Regulation
    \begin{compactitem}
    	\item Because I feel that I have to do it
    	\item Because I am supposed to do it
    	\item Because I don’t have any choice
    	\item Because it is something that I have to do
    \end{compactitem}
    \item Identified Regulation
    \begin{compactitem}
    	\item Because I believe that this game is important for me
    	\item It is for my own good
    	\item Because I think that this activity is good for me
    	\item Because I feel like playing this game
    \end{compactitem}
    \item Intrinsic Motivation
    \begin{compactitem}
    	\item Because I feel good when playing this game
    	\item Because this game is fun
    	\item Because I think that this activity is pleasant
    	\item Because I think that this game is interesting
    \end{compactitem}
\end{itemize}

\section{Motivated Strategies for Learning Questionnaire (MSLQ)} Responses were recorded on a 7-point semantic differential scale (1="Not at all true for me", 7="Very true for me") indicating how much the respondents agreed with a given statement. The instructions for responding to this questionnaire's items read: "The following questions ask about your \textbf{motivation for and attitudes about} practicing the Lunar Lander game. Remember \textit{there are no right or wrong answers}, just answer as accurately as possible. Use the scale below to answer the questions. If you think the statement is very true of you, place the slider at the rightmost position (Very true for me); if a statement is not at all true of you, place the slider at the leftmost position (Not at all true for me). If the statement is more or less true of you, place the slider somewhere in-between to best indicate how you feel."

\begin{itemize}
    \item Extrinsic Goal Orientation
    \begin{compactitem}
    	\item I want to do well in this game because it is important to show my ability to others
    	\item If I can, I want to get better scores in this game than most of the participants.
    	\item The most important thing for me is improving my overall score point average, so my main concern in this game is getting a good score
    	\item Getting a good score in this game is the most satisfying thing for me
    \end{compactitem}
    \item Task Value
    \begin{compactitem}
    	\item Understanding the purpose of this learning activity is important to me
    	\item I like this kind of game
    	\item I think learning to play this game is useful for me
    	\item I am very interested in this kind of game
    	\item It's important for me to learn to play this game
    	\item I think I will be able to use what I learn in this game in other situations
    \end{compactitem}
    \item Control of Learning Beliefs
    \begin{compactitem}
    	\item If I don't understand how to succeed in this game, it is because I didn't try hard enough
    	\item If I try hard enough, then I will understand how to succeed in the game
    	\item It is my own fault if I don't learn how to succeed in the game
    	\item If I learn in appropriate ways, then I will be able to succeed in the game
    \end{compactitem}
    \item Self-Efficacy for Learning and Performance
    \begin{compactitem}
    	\item I'm certain I can master skills this game teaches
    	\item I expect to do well in this game
    	\item I'm confident I can do an excellent job in this game
    	\item I'm confident I can master the most complex version of this game
    	\item I'm confident I can learn the basic skills this game requires
    	\item I'm certain I can play the most difficult mode in the game
    	\item I believe I will achieve good results in this game
    \end{compactitem}
\end{itemize}

\section{NASA Task Load Index (TLX)} Responses were recorded on a 20-point semantic differentiation scale (1="Very low", 2="Very high").

\begin{itemize}
    \item Frustration
    \begin{compactitem}
    	\item How insecure, discouraged, irritated, stressed, and annoyed were you while playing the game?
    \end{compactitem}
    \item Effort
    \begin{compactitem}
    	\item How hard did you have to work to perform at your level of performance in the game?
    \end{compactitem}
    \item Performance
    \begin{compactitem}
    	\item How successful were you in the game?
    \end{compactitem}
    \item Temporal demand
    \begin{compactitem}
    	\item How hurried or rushed was the pace of the game?
    \end{compactitem}
    \item Physical demand
    \begin{compactitem}
    	\item How physically demanding was the game?
    \end{compactitem}
    \item Mental demand
    \begin{compactitem}
    	\item How mentally demanding was the game?
    \end{compactitem}
\end{itemize}

\section{Improvement judgments} Responses were recorded on an 11-point semantic differentiation scale (1="Much worse", 11="Much better") and translated into scores ranging between -5 and 5 (0 indicating no improvement).
\begin{itemize}
    \item Within-session improvement
    \begin{compactitem}
        \item Rate how much you current level of performance has changed \textit{compared to the beginning of today's session}
    \end{compactitem}
    \item Improvement between consecutive sessions
    \begin{compactitem}
        \item Rate how much you current level of performance has changed \textit{compared to the previous session}
    \end{compactitem}
    \item Improvement between sessions 1 and 3
    \begin{compactitem}
        \item Rate how much you current level of performance has changed \textit{compared to the very first session of the experiment}
    \end{compactitem}
    \item Prospective improvement
    \begin{compactitem}
        \item Rate how much you expect to improve over the next session
    \end{compactitem}
\end{itemize}

\end{subappendices}